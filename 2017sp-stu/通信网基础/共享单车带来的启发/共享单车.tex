% 共享单车带来的启发
% LaTeX Template
% Version 1.0 
\documentclass[11pt]{ctexart} % 11pt即5号字体
\linespread{1.2} % 行距为1.2
\usepackage{url}  % 使用宏包url提供超链接
\usepackage[top=2cm, bottom=3cm, left=3cm, right=3cm]{geometry}  % 使用geometry来控制页面布局
\title{\centering \LARGE{共享单车带来的启发}\\% Title
} % Subtitle
\author{\textsc {陈李锋 14通信 2014081025 } % Author
} % Institution

%----------------------------------------------------------------------------------------

\begin{document}

\maketitle % Print the title section

\begin{abstract} %摘要
2017年一开学,共享单车忽然就变得火爆起来。上一次坐“小黄车”还是半年前在广州的大学城,一眨眼,汕头被各种五颜六色的共享单车给包围了,各个共享单车的大公司也拿到了好几轮融资。随着共享单车模式的火爆,一些其他的共享商品也开始催生,例如共享移动电源。进一步思考,共享单车等共享商品其实和共享经济概念是离不开的。本篇论文是从共享单车的模式开始,然后到共享商品,最后到共享经济的结构。
\\ \\关键词: 共享单车\quad  共享商品\quad 共享经济 
\\
 

\end{abstract}

\subsubsection*{\raggedleft{共享单车}}
\paragraph{}

共享单车是指企业与政府合作,在校园、地铁站点、公交站点、居民区、商业区、公共服务区等提供自行车单车共享服务,是共享经济的一种新形态。\\

共享单车的出现,已经越来越多地引起人们的注意。因为共享单车这个行业出现的时间不长,同时竞争对手比较多,每个商家都处于“摸着石头过河”的初级阶段。即使像nfo和摩拜这样大的企业c轮d论拿到一两亿美金的融资,但现在还是没有哪一家企业真正实现盈利的模式,这也就意味着这个行业还会重新洗牌重组。同时由于其符合低碳出行理念,政府从一开始处于善意观察期到现在积极支持,例如汕头的龙湖区就和摩拜单车签署了战略合作的协议,推动共享单车的使用。

科技可以改变出行,许多投资者都开始认可共享单车这种商业模式。创新工厂创始人李开复就非常认可这项技术,他对于现在共享单车还不盈利的市场表示是还没有规范化的结果。他认为只要共享单车的成本与收费能达成合适的比例,规模化后能产生价值。对于现状,许多人都是认为共享单车的市场也像之前的网约车那样出现滴滴合并Uber一家独大但最终盈利也不是很理想的情况。但李开复表示,滴滴盈利这么困难,因为人力成本高。共享单车相对互联网化,没有司机,容易产生对用户的价值。所以总的来说,共享单车这个市场刚刚起步,未来发展前景都很被看好。

\subsubsection*{\raggedleft{共享商品}}
\paragraph{}
  共享单车的火爆,提高了人们出门便捷性的同时,也提高了对共享商品的关注。共享商品指的是用户交纳一定的保证金或提供信用凭证,然后向企业提供比商品价格低得多的金钱换取商品一定时间的使用权。现在的共享商品不仅仅是共享单车、共享汽车,前一阵子也出现了共享移动电源等共享方式,下面简单介绍一下共享移动电源。
  
  共享移动电源,顾名思义就是提供公共移动电源的服务。其模式很类似共享单车,业务的基本形态是线下充电宝付押金租赁和在固定场景提供付费充电服务。在共享移动电源这个领域,主要是有“小电”和“Anker街电”这两家,其中,“街电” 宣布获得亿元级别的A轮融资。共享移动电源主要有两种移动充电解决方案,一种是Anker街电机柜模式,另外一种是小电的桌面型固定式模式,前者相对后者受众更广,因为其满足了移动电源便携性。
  共享移动电源本质上和共享单车没有本质上的区别,只是共享商品的形式不一样,这里就不多作解释。现在而言,共享单车是大势所趋,共享移动电源刚刚出来,其发展前景也一定程度上反映了共享商品的发展前景。对于共享商品的发展,科技的进步是占非常重要的因素,因为科技可以使过去的路和过去的壁垒都可能变得一钱不值,例如二维码与微信和支付宝很快就把许多线下支付设备替代掉了。所以,对于共享商品而言,相对于传统的商品经济,在科技的影响下,其发展可能更为迅猛和出乎意料。

\subsubsection*{\raggedleft{共享经济}}
\paragraph{}
共享经济(Sharing economy),又称分享型经济,是一种共用人力与资源的社会运作方式。它是指拥有闲置资源的机构或个人有偿让资源使用权给他人,让他人获取回报,分享者利用分享他人的闲置资源创造价值。

共享经济的出现,使未被充分利用的资源获得更有效的利用,从而使资源的整体利用效率变得更高。共享经济离不开共享商品,其商品所有者不仅仅是公司等机构,也可以是个人经济体。比如说,前者公司可以是摩拜等共享单车公司,后者可以是个人家庭通过Airbnb平台把自己家中的沙发租给“沙发客”。更有甚者,可以把开源OpenSourse 看为是共享经济的一种,因为即使软件的拥有权公开了,但商家是通过提过软件服务来获取利润的。

共享经济可以弱化商品的拥有权,强化商品的使用权,它可以让用户使用比较低的价格就可以享受高质量的服务。现在共享经济越来越火,人们更多地使用Airbnd共享房屋,人人office共享办公室,不仅仅是因为省钱,更多的是共享经济可以更好地满足他们各自独特的需求。例如Airbnd对于酒店,用户可以用同样的价格能获得更加舒适的居住,二来有的房东比较热情能够给旅行指点当地的情况。而第二个附加价值越来越开始显得重要,因为对于住酒店来说,他们提供的服务几乎是千篇一律的,但分享式服务跟现有服务的最大差异是打破服务的标准化,能够满足客户的细分需求。对客户而言付出去的钱所获得的体验更有价值,而对商家而言有更高效的平台找到契合特定需求的客户。

共享经济的基本特征有三个:借助网络作为信息平台、以闲置资源使用权的暂时性转移为本质、以物品的重复交易和高效利用为表现形式。共享经济的产生,是互联网和传统租凭业的结合。共享经济扩大了交易主体的可选择空间和福利提升空间,改变人们的产权观念,更重要的是改变了传统产业的运行环境,形成了一种新的供给模式和交易关系。

\subsubsection*{结语}
  共享单车的火爆,背后的原因是共享经济开始受到更多人们欢迎。共享经济作为互联网和传统租凭业结合的结果,可以很好地被投资家看好。其实许多发展很好的企业也是互联网与传统经济相结合的,例如淘宝等电子商务是互联网与传统商铺的结合,谷歌等搜索引擎是互联网与图书馆的结合。一个新的概念不会是完完全全都是前所未有全新的,它必然是传统产物的进化或改良。所以说,我们不仅仅需要学习互联网的知识,同时也应该去观察生活,用互联网和传统结合的方法,创造出更好的产品.
\\ \\ \\ 参考文献:

[1] 共享单车融资.新浪新闻. \url{ http//:tech.sina.com.cn/zl/post/detail/i/2017-01-19/pid_8509671.html} 

[2] 摩拜和龙湖区合作签约.汕头日报. \url{http://www.stlh.gov.cn/Gb/Home/Details.aspx?CateID=002010030&ID=26412}

[3] "街电”获得亿元级投资.投资潮网站. \url{http://www.investide.cn/news/344629}

[4] 共享经济. MBA 智库百科. \url{http://wiki.mbalib.com/wiki/共享经济}

[5] 共享经济. wiki百科. \url{ https://www.wikiwand.com/zh-cn/共享经济}



\end{document}
