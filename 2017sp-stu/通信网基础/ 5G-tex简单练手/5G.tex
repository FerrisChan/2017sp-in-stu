%% 5G 的简单介绍 
% LaTeX Template
% Version 1.0 

\documentclass[a4paper]{ctexart} 
\small
\linespread{1.2} % 行距为1.2
\title{\centering \LARGE{浅谈5G}\\% Title
} % Subtitle

\author{\textsc {陈李锋 14通信 2014081025 } % Author
} % Institution

%----------------------------------------------------------------------------------------

\begin{document}

\maketitle % Print the title section

\begin{abstract}
第五代移动通信系统(英语:5th generation mobile networks或5th generation wireless systems),简称5G,指的是移动通讯技术第五代,也是4G之后的延伸,目前正在积极研发中。相对于4G,5G的主要目标有5个,峰值速率更高,能效更高,频谱效率更高,用户体验速率更低,空口时延更短。其中前三个峰值速率,频谱速率和能效可以在原有4G的基础上改进,例如加深算法等。而用户的体验速率和空口延时则需要重新设计,需要更高的频率,更扁平的网络架构和新的帧结构。因为5G涉及的技术比较多,所以本文只是从这方面简单谈及5G的几个新型技术和应用前景。
\end{abstract}

\subsubsection*{\raggedleft{1. 5G关键技术之无线空口}}
\begin{enumerate}
\item 3D-MIMO
\paragraph{} 
\small 相对传统的2D-MIMO,3D-MIMO有着比8通道站更多的64通道新站,可以在水平和垂直两个维度进行信号方向的调整,可以使能量更加集中、方向更加准确,降低小区间和用户间的干扰,通过更多的空分,支持更多的用户在相同资源上并行传输,提升了小区的吞吐率。
\item 信道编码 Polar Code
\paragraph{}
\small 信道编码在通信传输中占比较重要的位置,在5G通信中,有三种方案,LDPC,Polar和增强型Turbo。其中,在国际无线标准化机构,中国华为主推PolarCode(极化码)方案,美国高通主推LDPC方案,法国主推Turbo2.0方案,最终短码方案由华为的极化码胜出。\\
极化码(Polar code)是一种前向错误更正编码方式,用于讯号传输,也是目前唯一能接近香农定理中香农极限的编码。极化码的构造核心是通过信道极化处理,在编码侧采用方法使各个子信道呈现出不同的可靠性,当码长持续增加时,部分信道将趋向于容量近于1的完美信道(无误码),另一部分信道趋向于容量接近于0的纯噪声信道,选择在容量接近于1的信道上直接传输信息以逼近信道容量。
\item 高频率极高频
\paragraph{}
\small 5G技术将可能使用的频谱是28GHz及60GHz,属极高频(EHF),比一般电讯业现行使用的频谱(如2.6GHz)高出许多。虽然5G能提供极快的传输速度,而且时延很低,但讯号的衍射能力(即绕过障碍物的能力)十分有限,且传送距离很短,这便需要增建更多基站以增加覆盖。
\end{enumerate}

\subsubsection*{\raggedleft{2. 5G带来的应用前景}}
\begin{enumerate}
\item 更快的上网速度

\paragraph{} 
\small 相对于4G,5G的峰值速度是4G的20倍,体验速度更是100倍。这么快的速率很容易就达到几Gbps,也就意味着人们上网的速度更快,需求变得更大更多。\\
 另外,更快的速度也将提升网络的带宽和容量,可以容纳更多的用户在同一时间登录网络。
 
\item 自动驾驶汽车
\paragraph{}
\small 自动驾驶汽车在这几年变得很火,因为除了传统的汽车制造厂例如特斯拉、宝马等,许多互联网公司例如谷歌,百度等都参与这个项目的研究。5G的提出,让自动驾驶真正成为可能。我们目前使用的4G网络,端到端时延的极限是50毫秒左右,还很难实现远程实时控制,但如果在5G时代,端到端的时延只需要1毫秒,足以满足智能交通乃至无人驾驶的要求。另外,因为5G工作的频率是极高频,工作覆盖范围比较小,所以需要的基站更多。也就让1平方公里内甚至可以同时有100万个网络连接成为可能,同时连接的设备更多,这些设备可以获知道路环境,提供行车信息,分析实时数据、智能预测路况,从而实现更加安全的自动驾驶。

\item 物联网
\paragraph{}
\small
物联网,是互联网、传统电信网等信息承载体,让所有能行使独立功能的普通物体实现互联互通的网络。其中,物联网也是计算机网络中的下一代网络的趋势。\\
目前的4G网络虽然可以提供较为理想的网速,但因其容量有限,并不足以支撑万物互联。5G网络容量的大幅度提升为实现“万物互联”提供了条件。物联网将现实世界数位化,应用范围十分广泛。物联网拉近分散的信息,统整物与物的数字信息,物联网的应用领域主要包括以下方面:运输和物流领域、健康医疗领域范围、智能环境(家庭、办公、工厂)领域、个人和社会领域等。可以看出,5G的出现,可以让物联网真正的实现
\end{enumerate}

\subsubsection*{结语}
\small 
  5G涉及的东西很多,现阶段的5G还处于一个起步阶段,相对于成熟技术实现更多的是通过不断测试和不同的技术方案博弈,共同制定一个标准。5G理论上真正商用可能在2020年前后,可以肯定的是,它的出现必定给各行各业带来新的变化,必定带来一个一个无线网络与每个人都息息相关的新时代。\\ \\ 参考文献:

%%%参考文献refelence
% \begin{thebibliography}{9}

%\bibitem{}
[1] 5G的wikipedia,https://www.wikiwand.com/zh-cn/5G
%\bibitem{}

%知乎关于5G的连接,https://www.zhihu.com/question/53878059
%\end{thebibliography}
\end{document}
