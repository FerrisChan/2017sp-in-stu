% !TEX TS-program = xelatex
% !TEX encoding = UTF-8 Unicode
% !Mode:: "TeX:UTF-8"

\documentclass{resume}
\usepackage{zh_CN-Adobefonts_external} % Simplified Chinese Support using external fonts (./fonts/zh_CN-Adobe/)
%\usepackage{zh_CN-Adobefonts_internal} % Simplified Chinese Support using system fonts
\usepackage{linespacing_fix} % disable extra space before next section
\usepackage{cite}

\begin{document}
\pagenumbering{gobble} % suppress displaying page number

\name{陈李锋 \normalsize\quad linux下c/c++开发——实习(可半年)}
\basicInfo{
  \email{ferris.chan@foxmail.com} \textperiodcentered\ 
  \phone{(+86) 13411982452} %\textperiodcentered\ 
   % QQ {995889520}
  }
 
\section{\faGraduationCap\  教育背景}

\datedsubsection{\textbf{汕头大学}}{\textit{专业}\ 14级通信工程}
主修课程:,c语言程序设计(91)嵌入式系统(86),数字电路设计(83),{}{}微机原理(81),计算机系统概论,通信网络基础等
\section{\faUsers\ 学习及主要项目经历}
\datedsubsection{\textbf{学习linux网络编程,C++等}}{2016 年7月 -- 至今}
% \role{linux,Python,latex}{个人学习}  %担任的角色
\begin{onehalfspacing}
\begin{itemize}
  \item 个人SHTTP的搭建
  
  在ubuntu16下,独自使用c语言构建和测试一个SHTTP的服务器端和客户端,
  
  实现了线程池等基本功能,增加了对linux网络编程的认识
  \item 搭建自己github代码库和文档,编写c++和mysql等代码和笔记
  \item 熟悉计算机网络自下而上架构,了解TCP/IP端口协议
  \item 寒假学习思科网上关于网络的英文课程,了解802基本协议等
  
\end{itemize}

\datedsubsection{\textbf{全国数学建模大赛}}{2015年7月 -- 2016年6月}
% \role{担任编程工作}{小组三人项目}  %担任的角色
\begin{onehalfspacing}
% 太阳影子定位问题和悬链线方程
\begin{itemize}
  \item 在一个半月内,搭建三人团队,使用Matlab语言等编程实现和优化城市最短路问题等,\\最终获得广东赛区二等奖
  \item 期间完成了多个项目建模编程,提高了编程能力和抗压能力
  \item 另外使我学习和加深对数据结构和许多算法的理解,例如图论和生成树等
\end{itemize}
\end{onehalfspacing}
\datedsubsection{\textbf{参与社团无线电和其他活动} }{2014年9月 -- 2015年6月}
% \role{干事}{团队综合项目}
\begin{itemize}
  \item 大一加入了无线电社团,学习和使用单片机、树莓派等,
  \item 另外期间义务帮助了十几位同学修理电脑,提高解决电脑问题能力,
  \item 期间也是校青协-绿丝带干事,参加了学校的“走向海洋”公益课程
\end{itemize}



\end{onehalfspacing}

% Reference Test
%\datedsubsection{\textbf{Paper Title\cite{zaharia2012resilient}}}{May. 2015}
%An xxx optimized for xxx\cite{verma2015large}
%\begin{itemize}
%  \item main contribution
%\end{itemize}
\section{\faHeartO\ 获奖情况}
  \datedline {全国数学建模比赛广东省二等奖(前20\%)}{ 2016 年8 月}
  \datedline {学业优秀奖学金(前30\%)}{2015-2016}
  
\section{\faCogs\ IT 技能}
% increase linespacing [parsep=0.5ex]
\begin{itemize}[parsep=0.5ex]
  \item 编程语言: C ==C++  > Python ==Matlab
  \item 平台: Linux
  \item 熟悉版本管理git、LaTeX等编程工具
\end{itemize}

\section{\faInfo\ 兴趣及其他}
% increase linespacing [parsep=0.5ex]
\begin{itemize}[parsep=0.5ex] 
  \item 英语 - 通过英语六级(486)
  \item GitHub: https://github.com/ferrischan
  \item 兴趣:夜跑、羽毛球,浏览技术论坛如V2ex,github等,
\end{itemize}

%% Reference
%\newpage
%\bibliographystyle{IEEETran}
%\bibliography{mycite}
\end{document}
