% !TEX TS-program = xelatex
% !TEX encoding = UTF-8 Unicode
% !Mode:: "TeX:UTF-8"

\documentclass{resume}
\usepackage{zh_CN-Adobefonts_external} % Simplified Chinese Support using external fonts (./fonts/zh_CN-Adobe/)
%\usepackage{zh_CN-Adobefonts_internal} % Simplified Chinese Support using system fonts
\usepackage{linespacing_fix} % disable extra space before next section
\usepackage{cite}

\begin{document}
\pagenumbering{gobble} % suppress displaying page number

\name{l陈李锋}

\basicInfo{
  \email{14lfchen@stu.edu.cn} \textperiodcentered\ 
  \phone{(+86) 13411982452} %\textperiodcentered\ 
  %\QQ {chen995889520}
  }
 
\section{\faGraduationCap\  教育背景}

\datedsubsection{\textbf{汕头大学}}{}
\textit{专业}\ 14通信工程

\section{\faUsers\ 学习及主要项目经历}
\datedsubsection{\textbf{大一 —— 学校绿丝带“绿满三月”活动} }{2014年9月 -- 2015年6月}
\role{干事}{团队综合项目}
\begin{itemize}
  \item 前期撰写活动策划书,组织活动内容等
  \item 中期户外活动,到中山公园与市民沟通参与问卷调查
  \item 后期回校整理问卷资料,开展评比活动等 
\end{itemize}

\datedsubsection{\textbf{大二——全国数学建模大赛}}{2015年7月 -- 2016年6月}
\role{担任编程工作}{小组三人项目}
\begin{onehalfspacing}
太阳影子定位问题和悬链线方程
\begin{itemize}
  \item 使用Matlab语言对影子视频进行影子长度分析
  \item 学习了数据结构和基本算法,例如图论和生成树等
\end{itemize}
\end{onehalfspacing}

\datedsubsection{\textbf{大三——学习linux网络编程}}{2016 年7月 -- 至今}
\role{linux,Python,latex}{个人学习}
\begin{onehalfspacing}
\begin{itemize}
  \item 了解linux下的C语言编程,如makefile,linux IO文件操作
  \item 熟悉linux环境工作,从开始使用centos,到完全转到ubuntu16.04
  \item 熟悉计算机网络,了解TCP/IP网络和socket编程
  \item 寒假参加思科网上英文课程,了解802基本协议
\end{itemize}
\end{onehalfspacing}

% Reference Test
%\datedsubsection{\textbf{Paper Title\cite{zaharia2012resilient}}}{May. 2015}
%An xxx optimized for xxx\cite{verma2015large}
%\begin{itemize}
%  \item main contribution
%\end{itemize}

\section{\faCogs\ IT 技能}
% increase linespacing [parsep=0.5ex]
\begin{itemize}[parsep=0.5ex]
  \item 编程语言: C ==matlab > Python 
  \item 平台: Linux
  \item 了解版本管理git和LaTeX使用
\end{itemize}

\section{\faHeartO\ 获奖情况}
\datedline{\textit{广东省二等奖},全国数学建模比赛}{2016 年8 月}
\datedline{学业优秀奖学金}{2015-2016}

\section{\faInfo\ 其他}
% increase linespacing [parsep=0.5ex]
\begin{itemize}[parsep=0.5ex]
  \item 英语 - 通过英语四六级
  \item GitHub: https://github.com/ferrischan

\end{itemize}

%% Reference
%\newpage
%\bibliographystyle{IEEETran}
%\bibliography{mycite}
\end{document}
